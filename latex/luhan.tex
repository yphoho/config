\documentclass[11pt, a4paper]{article}
%% \usepackage{amsmath}

%% for chinese font
\usepackage{fontspec}
\usepackage{xunicode}
\usepackage{xltxtra}
%% \setdefaultlanguage[babelshorthands]{zh-cn}

\usepackage[colorlinks=true, urlcolor=blue]{hyperref}
\usepackage[top=0.8in, bottom=1in, hmargin=1in]{geometry}

\setmainfont{WenQuanYi Micro Hei}
%% \setromanfont{Monaco}
\pagestyle{empty}       %% no page number

%% auto wrap in chinese
\XeTeXlinebreaklocale "zh"
\XeTeXlinebreakskip = 0pt plus 1pt

\hypersetup{
    %pdfstartview=Fit,
    pdfauthor={luhan},
    %% pdfsubject={resume},
    pdftitle={resume luhan},
    %% pdfkeywords={resume},
    %% pdfproducer={xetex},
    pdfcreator={luhan}
}



\begin{document}


\begin{center}
    \Huge{简{ }{ }历}       %% for many space
\end{center}





\section*{个人资料}         %% add * to no numbering

姓名:卢寒 \\
出生日期:1987-10 \\
性别:女 \\
学历:大学本科 \\
专业:计算机科学与技术 \\
电话:\mbox{13716835847} \\
% E-mail: \href{mailto:luhan1012@yeah.net}{luhan1012@yeah.net} \\
E-mail: luhan1012@yeah.net \\





\section*{专业能力}

熟悉 c,了解 c++/python; \\
熟悉 django 框架; \\
熟练掌握基本数据结构和算法; \\
熟悉 GDAL,OTB,libxml等开源库; \\
熟悉 ITK,VTK; \\
勤奋好学,自学能力强,有毅力和进取精神,责任心强。 \\






\section*{工作经历}

2011/05 -- 2012/09: \\
北京全景天地科技有限公司 | 研发中心 | 研发工程师 \\
主要负责 eCognition 的二次开发,规则集通用类库的设计,结合一些开源软件,根据客户需求进行系统定制。 \\
\\
2010/09 -- 2011/05: \\
河北省定州市清风店镇政府 | 干事 \\
主要负责人口基本信息的报表统计、分析和输出工作,以及虹膜采集系统的安装及使用培训工作。同时,做好相关政策文件的上传下达工作,及时应对突发事件。 \\





\section*{项目经验}

2012/06 -- 2012/08:医学影像三维重建 \\
开发语言:c/c++ \\
责任描述:读取 DICOM 医学影像数据,对多幅影像数据进行面绘制和体绘制。 \\
项目描述:医学影像数据可视化技术是科学计算可视化在医学领域的重要应用。在临床医学上,利用三维重建技术,可以将一系列二维断层图像重建成三维形体,为医生提供一个更加直观的模型。该项目采用了移动立方体法(面绘制的一种)和光线投射算法(体绘制的一种)对 DICOM 格式切片进行三维重建。 \\
\\
\\
2011/09 -- 2012/03:林火自动化遥感监测系统 \\
开发语言:c/c++ \\
责任描述:负责规则集的自动创建,底层对 eCognition Server 的调用。 \\
项目描述:项目主要用于林火的自动监测,提取过火区域、检测疑似火点。基于 ArcGIS、MapWinGIS、eCognition 进行联合开发,实现林火信息提取中用到的一些 GIS 功能。 \\
\\
2011/09 -- 2011/10:林科院项目 \\
开发语言:c/c++ \\
责任描述:解析 XML 文件,底层 eCognition 的调用编码。 \\
项目描述:调用 eCognition,对地学影像进行多尺度分割,并导出矢量,获取林区信息。 \\
\\
2011/06 -- 2011/07:BLV 插件开发 \\
开发语言:c/c++ \\
责任描述:模块编码和相关测试。 \\
项目描述:基于 Engine 中的算法框架,通过 eCognition SDK,将B、L、V三层的图像灰度值进行再计算,获得新图像。 \\
\\




\section*{在校实践经验}
2010/03 -- 2010/06:基于二进制粒子群算法求解 3-SAT 问题 \\
开发语言: c \\
项目描述:了解了基本粒子群算法、二进制粒子群算法、可满足性问题等,利用离散数学和算法设计中的相关知识,完成了 3-SAT 问题与整形多项式函数之间的转化,编写了二进制粒子群算法的模拟模块,随即 3-SAT 问题的产生模块,利用 BPSO 算法求解了较大规模的随机 3-SAT 问题,验证了基于 BPSO 算法解决 SAT 问题的有效性和可行性。 \\
\\



\section*{外语水平}
英语 | 六级 \\


\end{document}

