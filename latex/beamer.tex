\documentclass{beamer}
\usepackage[latin1]{inputenc}
\usetheme{Warsaw}
%\usetheme{Boadilla}
\title[Make a LaTeX presentation using Beamer]{Introduction  to Beamer\\How to make a presentation with LaTeX?}
\author{Nadir Soualem -- Astozzia}
\institute{Math-linux.com}
\date{Jule 13, 2007}


\begin{document}


%% titlepage
\begin{frame}
  \titlepage
\end{frame}

\begin{frame}{Introduction}
  This is a short introduction to Beamer class.
\end{frame}

%% columns
\begin{frame}
  \begin{columns}[c] % the "c" option specifies center vertical alignment
    \column{.5\textwidth} % column designated by a command
    Contents of the first column
    \column{.5\textwidth}
    Contents split \\ into two lines
  \end{columns}
\end{frame}

%% \begin{frame}
%%   \begin{columns}[t] % contents are top vertically aligned
%%     \begin{column}[5cm] % each column can also be its own environment
%%       Contents of first column \\ split into two lines
%%     \end{column}
%%     \begin{column}[T]{5cm} % alternative top-align that's better for graphics
%%       %% \includegraphics[height=3cm]{graphic.png}
%%     \end{column}
%%   \end{columns}
%% \end{frame}

%% blocks
\begin{frame}
  \begin{block}{Block Heading}
    Enlosing text in the ``block'' environment creates a distinct, headed block of text.
  \end{block}
  \begin{block}{Second Block Heading}
    This lets you visually distinguish parts of your slide easily.
  \end{block}
\end{frame}


%% revealing things incrementally
\begin{frame}
  Since I may want to focus on one item at a time in my presentation,
  \begin{itemize}
  \item I want to reveal only the first item on my list initially,
    \pause
  \item then the second item,
    \pause
  \item then the third,
    \pause
  \item and so on...
  \end{itemize}
\end{frame}


\end{document}

